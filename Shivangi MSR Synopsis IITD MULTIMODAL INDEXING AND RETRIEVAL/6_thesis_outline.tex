\section{Proposed Contents of the Thesis}
\label{sec:thesis_outline}
The proposed outline of the thesis is as follows:\\
\begin{enumerate}
\item \textbf{Chapter 1: Introduction}\\
In this chapter, we discuss the task of cross-modal retrieval and generation. We highlight the challenges involved in developing a cross-modal retrieval system and in the designing of evaluation metrics for the generative models. Next, we delve into the previous work and its limitations. Finally, we discuss our proposed solution and the major contribution of the work. \\

\item \textbf{Chapter 2: A Systematic Review of Cross-modal Retrieval Methods}\\
In this chapter, we review the available large-scale vision-language pre-trained models or foundational models from the of cross-modal retrieval perspective. We identified various issues in the existing methods and available benchmark datasets.\\

\item \textbf{Chapter 3: LCM: A Surprisingly Effective Framework for Supervised Cross-modal Retrieval}\\ In this chapter, we introduce a lightweight framework for solving the task of supervised cross-modal retrieval. We propose a novel 2-stage nearest neighbor search algorithm that can be used with any supervised representation learning method to provide state-of-the-art ranking and retrieval results. \\  

\item \textbf{Chapter 4: Joint versus Independent Framework for Supervised Cross-modal Retrieval}\\In this chapter, we compared the common representation learning method with the independent learning method for supervised cross-modal retrieval. We introduced a model training method called LM-I, which can parallel and independently learn representations of different modalities in a CMR system. We also introduce a new multi-labeled multimodal dataset, which includes modalities like image, text, and table data. \\

\item \textbf{Chapter 5: Evaluating Cross-modal Generative Models Using Retrieval Task}\\ We further propose a new application of cross-modal retrieval task in evaluating the cross-modal generative results. Our proposed methodology contrasts user-behavior-driven metrics for evaluating generative models with heuristics-driven metrics for evaluating the effectiveness of generative models on CMR tasks.\\

\item \textbf{Chapter 6: Conclusion and Future Work}\\
In this final chapter, we draw the thesis work to a close. Here, we concisely summarise the thesis content, highlight our key contributions, and address the limitations of the current work. By doing so, we set the stage for potential improvements and future work in the field.

\end{enumerate}

